\section{Warmup: Boolean Functions}
\label{sec:boolean-functions}

\begin{enumerate}
\item ~[3 points] Table \ref{tab:boolean-function-data-1} shows
  several data points (the $x$'s) along with corresponding labels
  ($y$). (That is, each row is an example with a label.) Write down
  three different Boolean functions, all of which can produce the
  label $y$ when given the inputs $x$.

  \begin{table}[h]
    \centering
    \begin{tabular}{ccccc}
      \toprule
      y & $x1$ & $x2$ & $x3$ & $x4$ \\
      \midrule
      0 & 0    & 1    & 1    & 0    \\
      0 & 1    & 1    & 1    & 0    \\
      1 & 0    & 1    & 1    & 1    \\
      \bottomrule
    \end{tabular}
    \caption{Initial data set}
    \label{tab:boolean-function-data-1}
  \end{table}
  
  Response:
  \begin{enumerate}
  	\item ~$y_1 = (x_2 \xor x_3) \; \lor x_4$
  	\item ~$y_2 = (x_2 \lor x_3) \; \land x_4$
  	\item ~$y_3 = x_4$
  \end{enumerate}
  
  
\item ~[5 points] Now the Table \ref{tab:boolean-function-data-1} is
  expanded to Table \ref{tab:boolean-function-data-2} by adding more
  data points. How many errors will each of your functions from the
  previous questions make on the expanded data set.

  \begin{table}[h]
    \centering
    \begin{tabular}{ccccc}
      \toprule
      $y$ & $x1$ & $x2$ & $x3$ & $x4$ \\
      \midrule
      0 & 0    & 1    & 1    & 0    \\
      0 & 1    & 1    & 1    & 0    \\
      1 & 0    & 1    & 1    & 1    \\
      1 & 1    & 0    & 1    & 1    \\
      0 & 0    & 1    & 1    & 0    \\
      1 & 1    & 1    & 0    & 1    \\
      \bottomrule
    \end{tabular}
    \caption{Expanded data set}
    \label{tab:boolean-function-data-2}
  \end{table}
  
  Response:
  \begin{enumerate}
  	\item ~The function $y_1$ has 0 errors given the expanded data set, since new features where the term $z = x2 \xor x3$ is always true when the output is true. The same is true for $z \lor x_4$.
  	\item ~The function $y_2$ has 0 errors given the expanded data set. It is clear that $y$ is true whenever $x_4$ is true, so the $\land$ operator only outputs true if the following conditions are met, at least one of $x_2$ or $x_3$ are true and $x_4$ is true.
 	\item ~The function $y_3$ offers the simplest solution since it reflects the value of $x_4$; furthermore, the function produces zero errors. 
  \end{enumerate}
\item ~[5 points] Is the function in Table
  \ref{tab:boolean-function-data-2} linearly separable? If so, write
  down a linear threshold function that classifies the data. If not,
  prove that there is no linear threshold function that can classify
  the data. \newline
	
  Response: The function that represents table 2 is linearly separable if the simplest solutions such as $y_3$ are considered. Replacing every $0$ label with $-1$ produces a function where the threshold is $y: x_4 >= 0$. 
\end{enumerate}

%%% Local Variables:
%%% mode: latex
%%% TeX-master: "hw2"
%%% End:
